% Options for packages loaded elsewhere
\PassOptionsToPackage{unicode}{hyperref}
\PassOptionsToPackage{hyphens}{url}
%
\documentclass[
]{article}
\usepackage{amsmath,amssymb}
\usepackage{iftex}
\ifPDFTeX
  \usepackage[T1]{fontenc}
  \usepackage[utf8]{inputenc}
  \usepackage{textcomp} % provide euro and other symbols
\else % if luatex or xetex
  \usepackage{unicode-math} % this also loads fontspec
  \defaultfontfeatures{Scale=MatchLowercase}
  \defaultfontfeatures[\rmfamily]{Ligatures=TeX,Scale=1}
\fi
\usepackage{lmodern}
\ifPDFTeX\else
  % xetex/luatex font selection
\fi
% Use upquote if available, for straight quotes in verbatim environments
\IfFileExists{upquote.sty}{\usepackage{upquote}}{}
\IfFileExists{microtype.sty}{% use microtype if available
  \usepackage[]{microtype}
  \UseMicrotypeSet[protrusion]{basicmath} % disable protrusion for tt fonts
}{}
\makeatletter
\@ifundefined{KOMAClassName}{% if non-KOMA class
  \IfFileExists{parskip.sty}{%
    \usepackage{parskip}
  }{% else
    \setlength{\parindent}{0pt}
    \setlength{\parskip}{6pt plus 2pt minus 1pt}}
}{% if KOMA class
  \KOMAoptions{parskip=half}}
\makeatother
\usepackage{xcolor}
\usepackage[margin=1in]{geometry}
\usepackage{graphicx}
\makeatletter
\def\maxwidth{\ifdim\Gin@nat@width>\linewidth\linewidth\else\Gin@nat@width\fi}
\def\maxheight{\ifdim\Gin@nat@height>\textheight\textheight\else\Gin@nat@height\fi}
\makeatother
% Scale images if necessary, so that they will not overflow the page
% margins by default, and it is still possible to overwrite the defaults
% using explicit options in \includegraphics[width, height, ...]{}
\setkeys{Gin}{width=\maxwidth,height=\maxheight,keepaspectratio}
% Set default figure placement to htbp
\makeatletter
\def\fps@figure{htbp}
\makeatother
\setlength{\emergencystretch}{3em} % prevent overfull lines
\providecommand{\tightlist}{%
  \setlength{\itemsep}{0pt}\setlength{\parskip}{0pt}}
\setcounter{secnumdepth}{-\maxdimen} % remove section numbering
\ifLuaTeX
  \usepackage{selnolig}  % disable illegal ligatures
\fi
\usepackage{bookmark}
\IfFileExists{xurl.sty}{\usepackage{xurl}}{} % add URL line breaks if available
\urlstyle{same}
\hypersetup{
  pdftitle={Project 1 Summary},
  pdfauthor={Cecilia Gonzales},
  hidelinks,
  pdfcreator={LaTeX via pandoc}}

\title{Project 1 Summary}
\author{Cecilia Gonzales}
\date{2025-07-01}

\begin{document}
\maketitle

\subsection{Project Contributors}\label{project-contributors}

\textbf{Ian McElveen}:

\textbf{Cecilia Gonzales}: Cecilia contributed to this project by doing
most of the analysis for questions 1, 2, and 4. This included doing the
data cleaning, exploratory analysis, model building, and interpretation
of results. She also wrote most of the project summary and helped make
the presentation.

\subsection{Project Overview}\label{project-overview}

\subsubsection{Question 1: Are changes in RSI related to team game
performance?}\label{question-1-are-changes-in-rsi-related-to-team-game-performance}

To address whether or not RSI is related to team game performance, our
approach took on a similar style to that of previous research. Past
research found that RSI was related to explosive in game performance and
used top in-game running speed of male D1 Basketball players as their
metric. They found that an increase in RSI from the beginning to the end
of practice the day before was associated with an in-game top running
speed that was greater than their median in-game top running speed for
the season. Since our data sets did not have in-game top running speed,
a combination of other explosiveness metrics(rebounds, steals, and
assists) were used as a proxy to top in-game speed. A game was
considered ``good'' or above median if all three of those metrics were
above their season median in a game. On the other hand, a game was
considered ``bad'' or below median if all three of those statistics were
below their season median in a game. These thresholds were purposefully
made hard to achieve to try to limit variability and concretely
distinguish between games that were above and below the season median.
Previous studies although, was able to measure each player's RSI before
and immediately after practice the day before each game. Our data was
not complete with RSI measurements before and after each practice so in
order to see how RSI changed for players and for the team, changes in
RSI were calculated to be the change since the last measurement taken.
This analysis built two models, one to predict whether the game would be
above median in all three metrics or not and another to predict whether
the game would be below all three metrics or not, both using the change
in RSI before the game as the only predictor. The models were built the
exact same way: a logistic regression model where the binary response
was whether or not the game was above median for the first model or
below median for the second model. The model that tried to predict
whether or not it would be a game with all three metrics above median,
it initially found that both the calculated intercept and coefficient
associated with the change in RSI were statistically significant at the
\(\alpha = 0.001\) significance level. This in itself is evidence that
changes in RSI are related to in game performance. This model had a
cross validated CER of 0.3977 meaning that it will make in incorrect
prediction around 39.77\% of the time when given new data. The true use
of the model was to see if changes in RSI would be able to predict if a
game would be considered above median in all metrics so we would like to
see how well it actually identified games that were considered above
median. For this, we looked at the sensitivity which is how well it
identified games that were considered above median. This model had a
sensitivity of 0.2358 meaning that given a game was above median in all
3 metrics, the model would predict that it would be above median about
23.58\% of the time. This suggests that under these conditions, changes
in RSI may not be sensitive enough to fully identify games that would be
considered ``good'' in this analysis. The model that tried to predict
whether or not it would be a game with all three metrics below median,
the intercept and the coefficient associated with the change in RSI as a
predictor for whether or not the game would be below median were both
statistically significant at the \(\alpha = 0.001\) significance level.
This suggests that, like for the first model, changes in RSI may have
some predictive power for games that were below median in rebounds,
steals, and assists. The model had a cross validated CER of 0.2046
meaning that the model would make an incorrect prediction around 20.46\%
of the time. But, just like before, the main metric that gives insight
into whether or not RSI can predict in game performance is the
sensitivity of the model. In this case it is 100\% meaning that given a
game was considered below median in all three metrics, the model will be
able to identify that that game will be below median by using change in
RSI as the only predictor almost perfectly. The main caveat of this
model is that the threshold for games to be considered a success or
below median was a probability of 0.05 or roughly 5\%. This means that
when the model produced probabilities for each game to be a success, the
most optimal threshold would be to consider all predictions above 0.05
as a game that would be below median. This is means that changes in RSI
don't make ``strong'' predictions for games that would be below median
but it is able to distinctly identify games that are below median and
those that are not. The false discovery rate is the rate at which the
model will be wrong given that it makes a prediction that an observation
is positive. In this case, it is 0.825 meaning that the model will be
wrong around 82.5\% of the time considering that it makes a positive
prediction. This means that with this lower threshold to be considered a
success or a below median game with this model, it is over predicting
the prevalence of below median games in the data. This could be due to a
lot of things such as confounding or the fact that for a game to be
considered ``bad'', it had to be below median in all three metrics. The
model may also be identifying games in which the team may have performed
poorly in two of the metrics but did not meet the threshold to be
considered ``bad'' in this analysis.

\subsubsection{Question 2: Are changes in RSI related to individual
statistic game
performance?}\label{question-2-are-changes-in-rsi-related-to-individual-statistic-game-performance}

A similar approach was taken to answer the question of how changes in
RSI are related in individual in-game performance. The same metrics were
used to determine whether a game was considered above or below median
but it was instead calculated for each individual player and their
median statistic values instead of the team as a whole for each game
throughout the season. For this analysis, four models were made. The
first two were pooled models that tried to predict whether a game was a
good game or not or a bad game or not for a player, treating all of the
players the same. The last two models were unpooled models and tried to
predict whether a game was good or not or bad or not by treating all of
the players differently and giving all of them unique slope and
intercept values for their logistic regression models.

\paragraph{Pooled Models}\label{pooled-models}

The first model built used a pooled method and tried to predict whether
or not a game would be above all three metrics for a player or not using
their change in RSI as the only predictor. This means that the model
treated all of the players the same and gave them all the same
coefficient and intercept value. This model did not find either the
intercept nor the coefficient associated with change in RSI to be
statistically significant. This model also had a cross validated CER of
0.4685 meaning that it will make an incorrect prediction around 46.85\%
of the time when looking at new data. This model though has a
sensitivity of 0.7964 meaning that it will correctly identify games that
are above median in all three statistics around 80\% of the time. Where
most of the error in this model comes from is the false discovery rate.
Given that this model makes a prediction that a game will be above
median, it will be wrong around 45.93\% of the time. This means that the
model could be identifying games that are are potentially above median
in 2 metrics but were not considered ``good'' in this analysis due to
the high threshold set in the beginning. For the second model, trying to
predict whether a game would be below median in all three metrics or
not, it was built using a pooled method where all of the players were
given the same intercept and slope coefficients. This model found that
only the intercept was statistically significant, not the slope
coefficient associated with change in RSI. It had a cross validated CER
of 0.2046 meaning that it would make an incorrect prediction around
20.46\% of the time. This model, similar to the model in the first
question has a lower threshold to be considered a success. The threshold
to be classified as a success for this model is 0.03. This suggests that
change in RSI, when used the exact same way for all players, does not
have enough predictive power for to determine whether or not a game will
be considered below median. This is supported through the sensitivity
and false discovery rates of the model. The sensitivity is 0.2354
meaning that it will correctly identify a game that is considered below
median only 23.54\% of the time. With this, given that the model
predicts a game will be below median, the model is wrong around 87.88\%
of the time. This suggests that changes in RSI when used the same way
for every athlete is a very poor predictor for in game performance.

\paragraph{Unpooled Models}\label{unpooled-models}

In order to account for the fact that RSI may be a better predictor for
in-game success than others, an unpooled model was created for both
above and below median games. With these models, we hoped to see if RSI
was a better predictor for in-game success for certain players and
identify who those players were. For the first model, it was built to
predict whether or not a game would be above median in all three metrics
for a player or not. This model gave each player their own slope and
intercept value. This model had a cross validated error of around
0.4627. But looking at specifically how the model was wrong, the model
was stronger than that of the pooled model built before. This model has
a sensitivity of 0.7054 suggesting that it correctly identifies games
that are above median around 71\% of the time. This model also has a
false discovery rate of 0.2867 meaning that given the model predicts
that a game will be considered above median, it will be wrong about 29\%
of the time. This improvement from the pooled model suggests that RSI
has different impacts on the players individually. Looking deeper into
the model, RSI had different predictive power for different athletes.
For the unpooled model used to predict whether a game would be below
median or not, the cross validated CER was 0.0382. While this is low,
it's more important to note that the sensitivity is 0.6471. This means
that it will correctly identify games that are considered below median
as being below median around 64.71\% of the time. With this, given that
the model predicts a game will be below median, it will be incorrect
around 83.58\% of the time. This model outperforms the pooled model in
every metric. This further suggests that changes in RSI have different
relationships with different athlete's in-game performance.

\subparagraph{Different Athlete
Impacts}\label{different-athlete-impacts}

For games that are considered above median, the athletes that had the
most impact when separated from the other athletes were ``ID\_42'',
``ID\_62'', ``ID\_66''. For these three players, RSI was considered to
be more impactful of a predictor when it comes to predicting whether or
not they will have a below median game. For games that were considered
below median, the athletes that had changes in RSI having the most
impact were ``ID\_40'' and ``ID\_42''.

\subsubsection{Question 3: Is the previous week's load related to
RSI?}\label{question-3-is-the-previous-weeks-load-related-to-rsi}

\subsubsection{Question 4: What is each athlete's variation in RSI? What
is a meaningful change in RSI for the team, and for the
athletes?}\label{question-4-what-is-each-athletes-variation-in-rsi-what-is-a-meaningful-change-in-rsi-for-the-team-and-for-the-athletes}

In order to quantify when RSI measurements were ``good'' or ``bad'',
they were compared to their mean and how many standard deviations away
from that mean that that measurement was. All of the player's RSI
measurements were checked and found to have no significant correlation
with time suggesting that no one's RSI measurement had a clear increase
or decrease with time. The same was checked for the team's averaged RSI
data and the same thing was found. This check insured that we weren't
missing any significant pattern in the data that may have been useful.
When looking at individual player's RSI measurements, we find that there
is not a clear trend for lows and highs for all players. In other words,
there were no clear points in the season in which all of the players had
noticeably high or low RSI measurements. That being said, when looking
at players individually, there tends to be a pattern where after a
measurement that is at least 1 standard deviation below the player's
mean RSI, there tended to be at least one more measurement that was
below 1 standard deviation following in the next 1 to 3 measurements.
This suggests that while RSI did not clearly have the same pattern for
all of the players, we can see that once a player dips somewhat below
their normal RSI range, they tend to have another low RSI measurement in
the following days to week. The same finding is true for measurements
that were at least 1 standard deviation above their mean. This suggests
that when a player dips or has a high RSI measurement, then they tend to
stay that way for a couple of days to a week. Another thing to note from
the plots for individual player's RSI is that even though all of the
players do not have a linear trend in their data (not monotonically
increasing or decreasing with time), players tend to have very different
distributions in their RSI measurements. When plotted on the same
y-axis, it's clear to see that some players have no overlap with each
other in their RSI measurements for the season. While it might be useful
to look at the RSI measurement themselves in future analysis, it might
also be useful to instead compare an individual RSI measurement to the
player's RSI measurement distribution knowing that some players just
have lower RSI measurements than others in general. Something
interesting that showed up in the plots was that there were no
measurements that were below 1 standard deviation for any of the players
after the start of February. Before then, there tended to be a
relatively consistent number of measurements that were above and below 1
standard deviation for players. There tended to be about 1 player who
was at least 1 standard deviation above their mean on a given day and
about 2 players who were at least 1 standard deviation below their mean.
But, after the start of February, there are no more players who
registered even 1 standard deviation below their mean and we see a sharp
increase in the number of players who are at least 1 standard deviation
above their mean. After the start of February, about 3-4 players tended
to have an RSI at least 1 standard deviation above their mean. We looked
for an underlying cause for this shift but nothing of significance was
found.

\subsection{Code Implementation}\label{code-implementation}

This project's code was written to run sequentially and runs best when
you run the questions in the order they were written. This analysis used
all of the tidyverse, readxl, aod, dplyr, ggplot2, lubridate, boot,
ROCR, purrr, ggforce, lme4, cv, caret, rsample,yardstick, corrplot, and
MASS libraries and used the ACWR-KinexonMBB, KinexonSessionMBB,
MBB-StatisticTrackingReport, VALD-DynamoMBB, and VALD-ForceDecksMBB data
sets. The first section of code is titled ``Data Cleaning'' and is where
we go through each data set and remove columns that are all filled with
NA. After removing irrelevant columns, only columns with important
information were chosen for each data set and the date column was
changed into a date class so that all were standardized across data sets
and could be used for plotting. If needed, once only important columns
remained, summary statistics and new columns with new information were
calculated for each data set. The rest of the analysis is broken down
into each question. Each section starts by filtering down whichever data
set will be used for that analysis into the relevant dates and players,
usually taking out players with not enough data and only using data from
the most recent Basketball season. After that, each section has some
plots with some preliminary exploratory analysis. For the first two
questions, lots of summary statistics needed to be calculated for each
date but also for each player at each date. In order to accomplish this,
loops and group\_by() methods were used to isolate each subgroup and
calculate the statistic needed. The first question mostly used
group\_by() but nested for loops were needed to accomplish the more
intricate calculations in the second question. When it came to building
and getting a preliminary estimate of the error for the models, all of
them were built using a training set and testing set made from the
entire data set(75\% to 25\% split from the data). When it came to
assessing the models built for questions 1 and 2, all of them were cross
validated either using cv.glm() or cv(). All of the plots made in this
analysis were built using the ggplot2 library.

\subsection{Final Results Summary}\label{final-results-summary}

\subsubsection{Question 1: Are changes in RSI related to team game
performance?}\label{question-1-are-changes-in-rsi-related-to-team-game-performance-1}

Changes in RSI when used to predict whether or not a game will be above
or below median or not were not entirely useful. In the way that this
analysis was done to limit variability and attempt mimic past research.
Changes in RSI do not seem useful when it came to predicting in game
success. The main caveat of this finding is that in order to limit
variability, the threshold to be considered ``good'' or ``bad'' was
purposefully set to be hard to achieve. Given this, using changes in RSI
as the only predictor and knowing that it was not calculated in the same
way as past research suggests that the findings in this analysis can't
completely be compared to those found in past research. It's also
important to note that in order to look at the team as a whole, change
in average RSI for the team was used. As found in Question 4, players do
not have similar changes in RSI throughout the season suggesting that
averaging removed potentially important patterns in the data. This
suggests that looking at changes in RSI at the player level may be more
impactful when looking into in-game success. To continue looking into
this question, it would be useful to look at this model but lower the
threshold for a game to be considered ``good'' or ``bad'' since the
threshold in this analysis was so high it meant that the models were not
nearly sensitive enough to recognize when all three metrics would be
above or below median.

\subsubsection{Question 2: Are changes in RSI related to individual
statistic game
performance?}\label{question-2-are-changes-in-rsi-related-to-individual-statistic-game-performance-1}

To see if there was a relationship between changes in RSI and in game
performance, the same approach from the previous question was used but
it used individual player RSI values and player median statistics
instead of averaged team RSI values and team summary statistics. Two
models were created to predict whether or not a game would be above
median in all three metrics for each player or not and whether or not a
game would be below median in all three metrics for each player or not.
These models found that when treating all of the players exactly the
same in the model that it did not have strong predictive power and could
not reliably identify when games would be above or below median or not.
Two more models were created to predict that same things. But, this time
each individual player was given their own intercept and slope value for
the logistic regression model. These models found that when
differentiating players in the model, they both had improved predictive
power from the models that treated all the players the same(pooled
models). This suggests that for identifying both above median and below
median games, differentiating players leads to improved accuracy within
the models. What this finding suggests is that change in RSI has a
different relationship with in game success for each player and
accounting for it leads to a better understanding of the underlying
relationship between change in RSI and in-game performance. The players
that had the greatest difference in slopes and intercepts for the model
used to predict if it would be above median in all three metrics were
``ID\_42'', ``ID\_62'', and ``ID\_66''. The players that had the most
significant difference in slopes and intercepts for the model used to
predict if it would be a below median game in all three metrics were
``ID\_40'' and ``ID\_42''. In short, I would mostly only use a jump in
RSI as an indicator of above median in-game play for ``ID\_42'',
``ID\_62'', and ``ID\_66'' and would only look at a dip in RSI as a
potential indicator of below median in game performance for ``ID\_40''
and ``ID\_42'' but would still take predictions with a grain of salt.

\subsubsection{Question 3:}\label{question-3}

\subsubsection{Question 4: What is each athlete's variation in RSI? What
is a meaningful change in RSI for the team, and for the
athletes?}\label{question-4-what-is-each-athletes-variation-in-rsi-what-is-a-meaningful-change-in-rsi-for-the-team-and-for-the-athletes-1}

Looking at the relationship that RSI had with time throughout the
season, there were no clear patterns in rises or falls in RSI for the
players. In order to quantify what was a ``good'', ``bad'' or ``normal''
RSI measurement for each player, each of their RSI measurements were
compared to their overall season distribution. A measurement was
considered ``good'' if it was above at least 1 standard deviation from
their mean and ``bad'' if it was below at least 1 standard deviation
from their mean and ``normal'' if it did not deviate far from the mean
in either direction. Looking at the trends that arise from the players
throughout the season, there are no clear rises or dips in RSI for the
team throughout the season. In fact, they seem to happen randomly for
players throughout the season where the number of players who have a
rise or dip in RSI stay consistent for each day measurements were taken.
This pattern only continues through the end of January though. Once
February starts, none of the players register a ``bad'' RSI measurement
for the rest of the season and we actually see a sharp spike in the
number of players who have ``good'' RSI measurements. Nothing in the
data we were given can explain this shift in RSI measurements. One
pattern that was found to be consistent for most of the players though
was that once a player registers a measurement that is one standard
deviation below their season mean, it is usually followed by another 1
or 2 measurements that are also at least 1 standard deviation below
their season mean in the following days to week. The same goes for
measurements that were above at least 1 standard deviation below the
mean. This suggests that when players have either dips or highs in RSI
they will stay there for a few days before returning to a more
``normal'' range with small fluctuations within their measurements. It's
also important to note that players tended to have very different RSI
distributions, sometimes not overlapping at all. Make sure to compare an
RSI measurement to a player's RSI distribution to get an understanding
of where they are at in terms of neuromuscular fatigue.

\subsection{Final Touches}\label{final-touches}

\end{document}
